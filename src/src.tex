\documentclass[UTF8]{ctexart}

\usepackage{titlesec}
\usepackage{titletoc}
\usepackage{graphicx}
\usepackage{caption}

\captionsetup[figure]{skip=0pt}
\captionsetup[table]{skip=0pt}

\title{基于神经网络回归模型的降雨量问题研究}

\author{李沐阳 \and 易领程 \and 钟绍恒}

\begin {document}

  \maketitle

  \newpage

  \tableofcontents

  \addcontentsline{tocloft}{section}{附录}

  \newpage

  \section{摘要}

本文针对局部地区的降雨量与其它气象相关的因素进行了建模和求解的通用算法的设计,运用线性回归,障碍计数,时间序列和神经网络等模型分别进行了尝试,并且使用遗传算法优化了模型的求解。

我们首先针对与降水量关联度最大的气压和气温等条件,将数据按照$30$天为一组进行平均处理整理后,根据气象学上的基本原理,运用线性回归模型拟合真实数据,完成了该简单模型的求解,取得了正确率$76 \%$的良好的预测效果,在这里同时使用了最小二乘法和遗传算法进行拟合,均能够取得很好的效果。在此时还尝试了障碍计数模型来应对未降雨的情况,但尝试后发现准确率极低,只有$10\%$左右,整理数据后分析得到数据的趋势具有连续变量的特征,大致更加符合回归模型这一类。

随后扩大研究的范围,添加了云层覆盖,湿度,风速为新的自变量,由于此时线性回归在多个自变量的影响下适应性较差,转变模型,根据深度学习的基本原理和神经网络中防止过拟合的技术,运用了神经网络回归模型进行拟合,仅使用$4$个神经层,共$200$多个神经元的情况下取得了良好的效果。同时,我们也再次展开了遗传算法的尝试,在仅有$4$层共$10$个神经元的网络中取得了$50\%$的正确率,提供了一种解决问题的新思路。

在数据具有明显季节性质,且相邻的日期可能相互影响的情况下,我们也尝试了时间序列模型,最后得到的效果也不如神经网络,原因可能是相互影响因素已经体现在了自变量中,经过分析,由于数据间并非存在明显的非线性关系,排除了决策树模型的可能性。尝试和分析过后,最后认为该神经网络模型更加准确。

该模型具有预测率较高,和可以轻易扩展到新的自变量的优点。生成该模型的方法可以轻易地被用于生成任何局部降雨模型,非常简单易行。在考虑到气象的周期性而忽略人为影响的情况下,本模型具有较为准确的预测能力。

关键词:\texttt{神经网络回归模型}  \texttt{线性回归模型}  \texttt{遗传算法}

  \newpage

  \section{问题重述}

  \newpage

  \section{模型假设和符号说明}

  \begin{table}[h]
    \centering
    \caption{符号说明}
    \begin{tabular}{p{6em}l}
      \hline 
      符号 & 说明 \\
      \hline 
      $x$ & 模型的自变量 \\
      $y$ & 模型的因变量 \\
      \hline
    \end{tabular}
  \end{table}

  \newpage

  \section{模型建立与求解}

  \begin{figure}[h]
    \centering
    \includegraphics[scale=0.1]{rust.jpg}
    \caption{test}
  \end{figure}

  \newpage

  \section{模型的优缺点与改进方法}

  \newpage

  \section{参考文献}

  \newpage

  \appendix
  \setcounter{secnumdepth}{-2} 
  \section{附录}

  \setcounter{secnumdepth}{3} 
  \subsection{源代码}

\end {document}
